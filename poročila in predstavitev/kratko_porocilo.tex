\documentclass[a4paper,12pt]{article}
\usepackage[slovene]{babel}
\usepackage[utf8]{inputenc}
\usepackage[T1]{fontenc}
\usepackage{lmodern}
\usepackage{amsmath}
\usepackage{hyperref} 
\usepackage{graphicx} 
\usepackage{amsfonts}
\usepackage{eurosym}
\usepackage{stackengine}
\usepackage{fancyhdr}
\usepackage{amssymb}
\usepackage{graphicx}
\usepackage{amsbsy}
\usepackage{mathtools}
\usepackage{amsthm}

\usepackage[output-decimal-marker={,}]{siunitx}

\begin{document}
\title{Kratka predstavitev projekta: Iterativni izračun Nashevega ravnotežja v matričnih igrah}
\author{Eva Babnik}
\date{April, 2022}
\maketitle


\newpage
\section{Nashevo ravnotežje in vrednost igre v matričnih igrah}

Spodobi se, da najprej v nekaj stavkih opišem Nashevo ravnotežje ter vrednost igre v matričnih igrah z ničelno vsoto za dva igralca. 
Matrično igro z dvema igralcema lahko predstavimo z matriko izplačil $A=(a_{ij})$, kjer prvi igralec izbere eno izmed $m$ vrstic, drugi pa hkrati izbere enega izmed
$n$ stolpcev. $A_{i\cdot}$ naj označuje $i$-to vrstico, $A_{\cdot j}$ pa $j$ - ti stolpec. Če igralca izbereta $i$-to vrstico in $j$-ti stolpec, potem drugi igralec plača prvemu igralcu $a_{ij}$. \par
Če prvi igralec izbere $i$-to vrstico z verjetnostjo $x_i$ in drugi izbere $j$-ti stolpec z verjetostjo $y_j$, pri čemer velja:
\begin{equation}
    \label{eqn:e1}
    x_i \geq 0,
\end{equation}
\begin{equation}
    \label{eqn:e2}
    \sum x_i = 1,
\end{equation}
\begin{equation}
    \label{eqn:e3}
y_i \geq 0,
\end{equation}
\begin{equation}
    \label{eqn:e4}
    \sum y_i = 1.
\end{equation}
Potem je pričakovano izplačilo prvemu igralcu $\sum \sum a_{ij}x_i y_j$. Poleg tega velja tudi:
\begin{equation*}
    \min_j \sum_i a_{ij}x_i \leq \max_i \sum_j a_{ij} y_j.
\end{equation*}
Trditev o minimaksu nam pove, da za neka vektorja verjetnosti $X = (x_1, \cdots, x_m)$ in $Y = (y_1, \cdots, y_n)$ v zgornji enačbi velja enakost. Tak par $(X^*, Y^*)$
se imenuje Nashevo ravnotežje. Vrednost igre $v$ pa je definirana kot:
\begin{equation*}
    v = \min_j \sum_{i} a_{ij}x_i = \max_i \sum_j a_{ij} y_j.
\end{equation*}
\section{Iterativno računanje rešitve igre}
V projektu bom implementirala več iterativnih algoritmov, ki nam vrnejo Nashevo ravnovesje in vrednost igre ter analizirala kako hitra je konvergenca posameznih metod.
Za implementacijo bom uporabila programski jezik \textit{Python}. Trenutno sem že implementirala dva algoritma, ki ju bom v nadaljevanju na kratko opisala. \par
\subsection{Metoda I}
Naj bo $V(t)$ vektor in $v_j(t)$ naj bo njegova $j$-ta komponenta. Označimo $\max V(t) = \max_j v_j(t)$ in $\min V(t) = \min_j v_j(t) $. Naj bo sistem $(U, V)$ sestavljen iz zaporedja $n$-dimenzionalnih vektorjev
$U(0), U(1), \cdots $ in zaporedja $m$-dimenzionalnih vektorjev $V(0), V(1), \cdots$ in naj velja $\min U(0) = \max V(0)$ in $U(t + 1) = U(t) + A_{i \cdot}$ ter $V(t+1) = V(t) + A_{\cdot j}$, pri čemer $i$ in $j$ zadoščata pogojem:
\begin{equation*}
    v_i(t) = \max V(t), \, \,  u_j(t) = \min U(t).
\end{equation*}
Potem za vrednost igre $v$ velja:
\begin{equation*}
\lim_{t \to \infty} \frac{\min U(t)}{t} = \lim_{t \to \infty} \frac{\max V(t)}{t} = v
\end{equation*}

Dokaz konvergence te trditve najdemo v \cite{vir1}.

\subsection{Metoda II}
Najprej uvedimo še nekaj nove notacije. Naj velja:
\begin{equation*}
    \begin{split}
    A_i = (a_{1i}, \cdots, a_{mi}, \underbrace{0, \cdots, 0}_{n - \text{komponent}}, -1), \\
    A_{0i} = (\underbrace{1, \cdots, 1}_{m-\text{komponent}}, \underbrace{0, \cdots, 0}_{n-\text{komponent}}, 0) \\
    \text{za} \, \, i = 1, \cdots, n
    \end{split}
\end{equation*} 
in 
\begin{equation*}
    \begin{split}
    A_i = (\underbrace{0, \cdots, 0}_{m - \text{komponent}},-a_{i-n, 1}, \cdots, -a_{i - n, n}, 1), \\
     A_{0i} = (\underbrace{0, \cdots, 0}_{m-\text{komponent}},\underbrace{1, \cdots, 1}_{n-\text{komponent}},  0) \\
    \text{za} \, \, i = n + 1, \cdots, m + n.
    \end{split}
\end{equation*} 
Definirajmo še vektor, ki predstavlja rešitev igre: $Z^* = (X^*, Y^*, v)$. $Z^*$ mora poleg \ref{eqn:e1}, \ref{eqn:e2}, \ref{eqn:e3}, \ref{eqn:e4}, ustrezati še pogoju:
\begin{equation}
\label{eqn:e5}
A_i \cdot Z^* \geq 0 \, \, \text{za} \, \, i = 1, \cdots, m + n.
\end{equation}
Metoda se začne s poljubnim vektorjem $Z^{(1)}$, ki zadošča \ref{eqn:e1}, \ref{eqn:e2}, \ref{eqn:e3} in \ref{eqn:e4}. Sedaj predpostavimo,
 da smo prišli na $k$-ti korak
iteracije, in dobili vektor $Z^{(k)}$,  ki ustreza \ref{eqn:e1}, \ref{eqn:e2}, \ref{eqn:e3} in \ref{eqn:e4}. Če velja tudi
\ref{eqn:e5}, je $Z^{(k)}$ rešitev igre in smo zaključili. Sicer pa naj bo $j_k$ tak indeks, da bo veljalo
$A_{j_k} \cdot Z^{(k)} \leq A_i \cdot Z^{(k)}$ za vse $i = 1, \cdots, m + n$. Če obstaja več takih indeksov, lahko poljubno izberemo.
 Če torej poznamo indeks $j_k$, lahko
dobimo nov vektor $\bar{Z}^{(k+1)} = (\bar{X}^{(k+1)}, \bar{Y}^{(k+1)}, \bar{v}^{(k+1)})$ na sledeči način:
\begin{equation*}
    \bar{Z}^{(k+1)} = Z^{(k)} + \alpha B_{j_k} + \beta B_{0j_k},
\end{equation*}
kjer je
\begin{equation*}
    \alpha = - Z^{(k)} \cdot B_{j_k} [1 - \cos^2{\theta_{j_k}}]^{-1},
\end{equation*}
\begin{equation*}
    \beta = b_{0j_k} - [Z^{(k)} + \alpha B_{j_k}] \cdot B_{0j_k},
\end{equation*}
\begin{equation*}
    b_{0j_k} = \frac{1}{(A_{0j_k}\cdot A_{0j_k})^{1/2}},
\end{equation*}
\begin{equation*}
B_{j_k} = \frac{A_{j_k}}{(A_{j_k} \cdot A_{j_k})^{1/2}}, 
\end{equation*}
\begin{equation*}
B_{0j_k} = \frac{A_{0j_k}}{(A_{0j_k}\cdot A_{0j_k})^{1/2}}
\end{equation*}
in 
\begin{equation*}
 \cos{\theta_{j_k}} = \frac{A_{0j_k} \cdot A_{j_k}}{(A_{0j_k}\cdot A_{0j_k})^{1/2}(A_{j_k}\cdot A_{j_k})^{1/2}}.
\end{equation*}
Sedaj predpostavimo, da velja $j_k < (n +1)$. (V primeru, da bi bil $j_k \geq n + 1$, bi komponente $x$ ostale nespremenjene, postopek, opisan v nadaljevanju, pa bi veljal za $y$ komponente.)
Če $\bar{Z}^{k+1}$ ustreza \ref{eqn:e5}, potem nastavimo $Z^{k+1} = \bar{Z}^{k+1}$,
v nasprotnem primeru pa moramo, da dobimo $Z^{k+1}$, narediti še nekaj korakov. Najprej vse negativne $x$-komponente vektorja $\bar{Z}^{k+1}$ nastavimo na 0.
Predpostavimo, da so $\bar{x}_1^{(k+1)}, \cdots, \bar{x}_r^{(k+1)}, \, \, r < m$ negativne komponente vektorja $\bar{Z}^{(k+1)}$. Nato 
izračunamo vse vsote $\bar{x}_i^{(k+1)} + \frac{\sum_{i=1}^r \bar{x}_i^{(k+1)}}{m - r} $ za $i = r +1, \cdots, m$. Za vsak tak $i$, za katerega
je vsota negativna, nastavimo $x_i^{(k+1)} = 0$. Če nobena vsota ni negativna, lahko tvorimo preostanek vektorja $Z^{(k+1)}$. Spet predpostavimo, da so nekatere vsote za $i = r+1, \cdots, r +s$ negativne. Ponovno izračunamo vsote 
$\bar{x}_i^{(k+1)} + \frac{\sum_{i=1}^{r+s} \bar{x}_i^{(k+1)}}{m - (r+s)} $ za $i = r + s, \cdots, m$. Če nobena vsota ni negativna, tvorimo preostanek vektorja $Z^{(k+1)}$, sicer pa ponavljamo zgornji postopek, dokler nobena od vsot ni negativna.\par
Predpostavimo, da za $i = 1, \cdots, t$ velja, da je $\bar{x}_i^{(k+1)} \leq 0$ ali pa, da je $\bar{x}_i^{(k+1)}$ tak, da je zanj katera od zgoraj definiranih vsot negativna. Potem lahko vektor $Z^{(k+1)}$
tvorimo na sledeči način:
\begin{equation*}
    x_1^{(k+1)} = \cdots = x_t^{(k+1)} = 0,
\end{equation*}  
\begin{equation*}
    x_i^{(k+1)} = \bar{x}^{(k+1)} +  \frac{\sum_{i=1}^{t}\bar{x}_i^{(k+1)}}{m-t} \, \, \text{za} \, \, i = t+1, \cdots, m,
\end{equation*}
\begin{equation*}
    y_j^{(k+1)} = \bar{y}_j^{(k+1)} \, \, \text{za} \, \, j = 1, \cdots, n,
\end{equation*}
\begin{equation*}
    v^{(k+1)} = \bar{v}^{(k+1)}.
\end{equation*}

Podrobnejšo izpeljavo algoritma in dokaz konvergence najdemo v \cite{vir2}.
\section{Načrt nadaljnega dela}
V nadaljevanju želim implementirati še kakšen iterativen algoritem ter algortime nato med seboj primerjati ter analizirati, kako hitra 
je konvergenca posameznih metod. Cilj je najti najboljši iterativni algoritem, ki nam vrne optimalno strategijo in vrednost matrične igre. 
 
\begin{thebibliography}{99}
\bibitem{vir1}
    J.~Robinson, \emph{An Iterative Method of Solving a Game},  Annals of Mathematics, \textbf{1951}  strani od 296 do 301. Dostopno na: https://www.jstor.org/stable/1969530
\bibitem{vir2}
  R. J.~Jirka, \emph{An iterative method for finding a solution to a zero-sum two person rectangular game}, \textbf{1959}.
\end{thebibliography}
    \end{document}
